\documentclass{article}
\begin{document}
In what follows, we proposed a series of experiments to study effects of mitigation and vaccination based 
on the population characteristics in Leon County. We will study dispersed or concentrated interventions: dispersed or concentrated. A dispersed intervention will apply uniformly across a single environment (whether homes, schools, or workplaces), while a concentrated intervention will apply to selected schools or workplaces. We study three different mitigation strategies with the objectivee of minimizing the overall number of infections in Leon County. 
Our SIR model assumes that all infectuous individuals eventually recover, and thus cannot be reinfected. The recovered groups break down further into those who remain asymptomatic, those who exhibit symptoms, those who are hospitalized but are released, and the people who die. Our model does not take these different groups into account. 

\subsection{Dispersed Interventions}
We assume that the entire population is composed of individuals, some wearing masks, and some social distancing. 
An individual with a mask might or might not social distance, while an individual who social distances, might or might not wear a mask. 

To describe the dispersed experiments that follow, we define the fraction of people in schools and the workplace 
wearing masks, $(\mu_s$ and $\mu_w)$, and similarly, the fraction of people in schools and the workplace social distancing, $\delta_s$ and $\delta_w$. Two other parameters are the mask and distance reduction effects, $\mu_r$ and $\delta_r$, which lie in the range $[0,1]$.

Each edge of the network graph has a weight $w$, set to unity by default. The effects of masking and social distancing are incorporated into the model by modifying the edge weight. 

\subsubsection{Experiment 1}
Set $\mu_s=\mu_w=\delta_s=\delta_w$ to $p$. Thus, $100p$ percent
of the population in schools and in the workplaces are wearing masks and are social distancing. The school population with masks satisfies a Bernouilli distribution with probability $p=\mu_s$, and similarly for all people in the workplace, who where masks according to $\textrm{Bern}(\mu_w)$. Social distancing is treated the same way. The school and workplace population social distances according to $\textrm{Bern}(\delta_s)$ and $\textrm{Bern}(\delta_w)$ respectively. We run two experiments, one with $p=0.3$, and one with $p=0.7$. For each case, we run the simulation for $\mu_r\in [0.,0.2,0.4,0.6,0.8,1.0]$ and the same values for $\delta_r$. 

Each person $P_n$ within an environment either wears a mask or not. An edge corresponds to the interaction between two people $P_n$ and $P_m$ in the same environment $(m\neq n)$. Let $\sigma_{e,n}$ equal 0 or 1, depending on whether a person $n$ in environment $e$ is  a mask or not. The environments $e=\{s,h,e\}$ can be school, home, or workplace. The weight is is reduced by the factor $(1-\mu_r\sigma_{e,n})(1-\mu_r\sigma_{e,m})$. The effect of distancing further reduces the edge weight by the factor $(1-\delta_r\delta_{e,n,m})$ where $\delta_{n,m}$ is 0 or 1, depending on whether the two people forming the edge in environment $e$ are social distancing. 


\subsection{Concentrated Interventions}
Another approach to examining the effect of masks is to rank the schools and workplaces from largest to smallest, and stipulate masking is mandatory in the largest business and schools. We will consider that p\% of the population is wearing masks of effectiveness $\mu_r$. This p\% will be taken from first from the largest businesses, followed by the next largest business, and so on down the line. At some point, the business that remains will have more people than the allocation. The masks will be warn by a random selection of employees. [NEED TO EXPLAIN BETTER. WITH MATH? 

We will repeat experiment 1 by allocating masks and imposing social distancing to the top $n_w$ businesses. 
Then repeat the experiment with the top $n_s$ largest schools. We will consider 100\% efficacy of masks (i.e., reducation of 100\%, which removes all edge weights from the appropriate environment. Thus, if fully effective masks are worn by all employees of a particular business, they edge weights are set to zero for that business. 
Note that this is not equivalent to shutting the businesses, since a shutdown would relegate the employees to the homes for longer periods of time, increasing the risk of contamination if other household members hold jobs in businesses not shut down or have children in the schools. 

\subsection{Allocation of vaccines}
We will make assumption on the properties of vaccines, given the constraints of our model. If a vaccine is fully effective, once given a vaccine dose, a person becomes effectively recovered. They are no longer able to transmit the disease. In what follows, $N_v$ is the total number of available vaccines. How to allocate them is described in the three strategies below. 
\begin{enumerate}
\item Before the simulation starts, we randomly select $N_v$ people from the population and vaccinate them, effectivelely changing their status to recovered. We will assume that nobody is social distancing or wearing masks. 
\item A second approach will be to allocate 70\% of the children (chosen randomly) starting with the largest schools and working our way down, until $N_v$ vaccinations are allocated. 
\item We will repeat the ranked approaches for workplaces. 
\item
First, we vaccinate  0\%, 30\% , 70\%, and 100\% of the entire population. Then I will vaccinate different age groups (what age groups would you recommend? Perhaps 20-30, 30-40, 40-50, 50-60, 60-70, 70 and above? I would vaccinate a fixed percentage of a given age group. Then I will do what we discussed: vaccination of the largest 10 businesses, the largest 30 businesses, the largest 100 businesses. I would vaccinate 0\%, 30\%, 70\%, and 100\% of these businesses. Then do the same experiment on schools. All these experiments will be a good start .
\end{enumerate}

\subsection{Concentrated Interventions}
1.	Concentrated or dispersed interventions

a.	Concentrate percentage of people masked or distanced to certain workplaces and schools or disperse them randomly

i.	Large employers policies can affect lots of people

ii.	Difficult to enforce policies on lots of small employers

iii.	Should amplify social network effects?

iv.	PPE might have limited availability.

v.	Some business can socially distant, some cant. (this might be able to be separate item to look at)

b.	Modify the percentage of people wearing masks

i.	10, 25, 50, 75

c.	Modify the percentage of people distancing

i.	10, 25, 50, 75

d.	Conditions:

i.	Random spread of masking—workplace

ii.	Random spread of masking—schools

iii.	Random spread of distance—workplace

iv.	Random spread of distance—schools

v.	Give masks to workplaces largest first

vi.	Give masks to workplaces smallest first

vii.	Give masks to schools largest first

viii.	Give masks to schools smallest first

ix.	(I am leaving out possible runs of the simulation)

e.	Figures

i.	Social network graph showing random masking on nodes and distancing on edges

ii.	Create heatmap

1.	Plot of each condition above

2.	Subplot based on masking and distancing

3.	Color of square:

a.	Maximum infected percentage

b.	Date at maximum infected percentage

c.	Maximum recovered percentage

d.	Date at maximum recovered percentage
\end{document}
